\documentclass{beamer}

\usetheme{femto}
\usepackage{media9}
\usepackage{animate}
\usepackage{tcolorbox}
\usepackage{mathtools}
%\graphicspath{{Figures/}{.}}
%\usepackage{graphics}
\graphicspath{ 
{square/}
{parallelhand/}
}

\title{\normalsize{Parallel robots.}}

\subtitle{\small{UBFC (15th March 2023)}}

\author{Grover Aruquipa\\
Supervision: Redwan Dahmouche}


\begin{document}

\begin{frame}[plain,c]
\titlepage
\end{frame}


%%%%%%%%%%%%%%%%%%%%%%%%%%%%%%%%%%%%%%%%%%%%%%%%%


%%%%%%%%%%%%%%%%%%%%%%%%%%%%%%%%%%%%%%%%%%%%%%%%%



%%%%%%%%%%%%%%%%%%%%%%%%%%%%%%%%%%%%%%%%%%%%%%%%%

%\begin{frame}{\twistshape Introduction}
%HASEL actuator
%\begin{columns}
%\begin{column}{0.5\textwidth}
%\hfill
%\begin{itemize}
%  \item The stiffness is smaller than DEAs %of the same dimensions and the same solid dielectric material, because part of the dielectric layer is liquid instead of solid.
%  \item Higher actuation strains than DEAs at the same voltage. 
%  \item Low power consummation.
%  \item Easy fabrication procedure.
%\end{itemize}
%  \end{column}
%\begin{column}{0.5\textwidth}
%\hfill
%  \includegraphics[width=1\linewidth,height=4cm]{HASEL_actuator.png}
%  \end{column}
%\end{columns}
%\end{frame}



%%%%%%%%%%%%%%%%%%%%%%%%%%%%%%%%%%%%%%%%%%%%%%%%%

%\begin{frame}{\twistshape Introduction}
%Experimental setup
%\vspace{-0.5cm}
%\centering
%\includegraphics[width=0.7\linewidth,height=5cm]{experimental_setup.png}
%\end{frame}




%%%%%%%%%%%%%%%%%%%%%%%%%%%%%%%%%%%%%%%%%%%%%%%%%%

%%%%%%%%%%%%%%%%%%%%%%%%%%%%%%%%%%%%%%%%%%%%%%%%%



%%%%%%%%%%%%%%%%%%%%%%%%%%%%%%%%%%%%%%%%%%%%%%%%%




%%%%%%%%%%%%%%%%%%%%%%%%%%%%%%%%%%%%%%%%%%%%%%%%%

%%%%%%%%%%%%%%%%%%%%%%%%%%%%%%%%%%%%%%%%%%%%%%%%%%%%%%%%%%%%%%%%%55



%%%%%%%%%%%%%%%%%%%%%%%%%%%%%%%%%%%%%%%%%%%%%%%%%%%%%


%%%%%%%%%%%%%%%%%%%%%%%%%%%%%%%%%%%%%%%%%%%%%%%%%%%%%





%%%%%%%%%%%%%%%%%%%%%%%%%%%%%%%%%%%%%%%%%%%%%%%%%%%%%



%%%%%%%%%%%%%%%%%%%%%%%%%%%%%%%%%%%%%%%%%%%%%%%%%%

\begin{frame}{Parallel Robot}
%\vspace{0.5cm}

Square Movement
\vspace{0.4cm}

\centering
\animategraphics[loop ,autoplay ,scale=0.1, angle=0]{10}{square_}{1}{52}

\end{frame}

\begin{frame}{Application 1}
%\vspace{0.5cm}
Continuos Rotation
\vspace{0.4cm}

\centering
\animategraphics[loop ,autoplay ,scale=0.15, angle=0]{10}{parallelhand_}{1}{52}
\end{frame}



\end{document}



%%%%%%%%%%%%%%%%%%%%%%%%%%%%%%%%%%%%%%%%%%%%%%%%%
